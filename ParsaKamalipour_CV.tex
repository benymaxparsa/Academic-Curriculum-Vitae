
%% start of file `template.tex'.
%% Copyright 2006-2015 Xavier Danaux (xdanaux@gmail.com), 2020-2021 moderncv maintainers (github.com/moderncv).
%
% This work may be distributed and/or modified under the
% conditions of the LaTeX Project Public License version 1.3c,
% available at http://www.latex-project.org/lppl/.

\newcommand{\CVNote}{\today}

\documentclass[10pt,a4paper,sans]{moderncv}        % possible options include font size ('10pt', '11pt' and '12pt'), paper size ('a4paper', 'letterpaper', 'a5paper', 'legalpaper', 'executivepaper' and 'landscape') and font family ('sans' and 'roman')

% moderncv themes
\moderncvstyle{banking}                             % style options are 'casual' (default), 'classic', 'banking', 'oldstyle' and 'fancy'
\moderncvcolor{burgundy}                               % color options 'black', 'blue' (default), 'burgundy', 'green', 'grey', 'orange', 'purple' and 'red'
%\renewcommand{\familydefault}{\sfdefault}         % to set the default font; use '\sfdefault' for the default sans serif font, '\rmdefault' for the default roman one, or any tex font name
%\nopagenumbers{}                                  % uncomment to suppress automatic page numbering for CVs longer than one page

% adjust the page margins
\usepackage[scale=0.85]{geometry}
% \usepackage[scale=0.77]{geometry}
%\setlength{\footskip}{136.00005pt}                 % depending on the amount of information in the footer, you need to change this value. comment this line out and set it to the size given in the warning
%\setlength{\hintscolumnwidth}{3cm}                % if you want to change the width of the column with the dates
%\setlength{\makecvheadnamewidth}{10cm}            % for the 'classic' style, if you want to force the width allocated to your name and avoid line breaks. be careful though, the length is normally calculated to avoid any overlap with your personal info; use this at your own typographical risks...

%\setlength{\footskip}{136.00005pt}

% font loading
% for luatex and xetex, do not use inputenc and fontenc
% see https://tex.stackexchange.com/a/496643
\ifxetexorluatex
\usepackage{fontspec}
\usepackage{unicode-math}
\defaultfontfeatures{Ligatures=TeX}
\setmainfont{Latin Modern Roman}
\setsansfont{Latin Modern Sans}
\setmonofont{Latin Modern Mono}
\setmathfont{Latin Modern Math}
\else
\usepackage[utf8]{inputenc}
\usepackage[T1]{fontenc}
\usepackage{lmodern}
\fi
\usepackage{MnSymbol}
% document language
\usepackage[english]{babel}  % FIXME: using spanish breaks moderncv
\usepackage{multicol}
\newcommand{\ts}{\textsuperscript}
% personal data
\name{Parsa}{Kamalipour}
% \title{CV}                               % optional, remove / comment the line if not wanted
%\born{4 July 1776}                                 % optional, remove / comment the line if not wanted
\address{Montréal, QC}{Canada}% optional, remove / comment the line if not wanted; the "postcode city" and "country" arguments can be omitted or provided empty
%\phone[mobile]{+98~(939)~198~5636}                     % optional, remove / comment the line if not wanted; the optional "type" of the phone can be "mobile" (default), "fixed" or "fax"
%\phone[fixed]{+2~(345)~678~901}
%\phone[fax]{+3~(456)~789~012}
\email{parsakamalipour.edu@gmail.com}                               % optional, remove / comment the line if not wanted
\homepage{benymaxparsa.github.io}                         % optional, remove / comment the line if not wanted

% Social icons
\social[linkedin]{parsakamalipour}                        % optional, remove / comment the line if not wanted
%\social[xing]{john\_doe}                           % optional, remove / comment the line if not wanted
%\social[twitter]{ji\_doe}                             % optional, remove / comment the line if not wanted
\social[github]{benymaxparsa}                              % optional, remove / comment the line if not wanted
%\social[gitlab]{jdoe}                              % optional, remove / comment the line if not wanted
%\social[stackoverflow]{0000000/johndoe}            % optional, remove / comment the line if not wanted
%\social[bitbucket]{jdoe}                           % optional, remove / comment the line if not wanted
%\social[skype]{benymax.parsa}                               % optional, remove / comment the line if not wanted
%\social[orcid]{0000-0003-2546-9676}                  % optional, remove / comment the line if not wanted
%\social[researchgate]{Parsa-Kamalipour}                        % optional, remove / comment the line if not wanted
%\social[researcherid]{jdoe}                        % optional, remove / comment the line if not wanted
%\social[telegram]{BenyMaxParsa}                            % optional, remove / comment the line if not wanted
%\social[whatsapp]{12345678901}                     % optional, remove / comment the line if not wanted
%\social[signal]{12345678901}                       % optional, remove / comment the line if not wanted
%\social[matrix]{@johndoe:matrix.org}               % optional, remove / comment the line if not wanted
\social[googlescholar]{eBNZsM0AAAAJ}            % optional, remove / comment the line if not wanted


%\extrainfo{additional information}                 % optional, remove / comment the line if not wanted
%\photo[64pt][0.4pt]{picture}                       % optional, remove / comment the line if not wanted; '64pt' is the height the picture must be resized to, 0.4pt is the thickness of the frame around it (put it to 0pt for no frame) and 'picture' is the name of the picture file
%\quote{Some quote}                                 % optional, remove / comment the line if not wanted

% bibliography adjustments (only useful if you make citations in your resume, or print a list of publications using BibTeX)
%   to show numerical labels in the bibliography (default is to show no labels)
%\makeatletter\renewcommand*{\bibliographyitemlabel}{\@biblabel{\arabic{enumiv}}}\makeatother
\renewcommand*{\bibliographyitemlabel}{[\arabic{enumiv}]}
%   to redefine the bibliography heading string ("Publications")
%\renewcommand{\refname}{Articles}

\usepackage{tikz}

\newcommand{\ExternalLink}{%
	\tikz[x=1.2ex, y=1.2ex, baseline=-0.05ex]{%
		\begin{scope}[x=1ex, y=1ex]
			\clip (-0.1,-0.1)
			--++ (-0, 1.2)
			--++ (0.6, 0)
			--++ (0, -0.6)
			--++ (0.6, 0)
			--++ (0, -1);
			\path[draw,
			line width = 0.5,
			rounded corners=0.5]
			(0,0) rectangle (1,1);
		\end{scope}
		\path[draw, line width = 0.5] (0.5, 0.5)
		-- (1, 1);
		\path[draw, line width = 0.5] (0.6, 1)
		-- (1, 1) -- (1, 0.6);
	}
}

% bibliography with mutiple entries
%\usepackage{multibib}
%\newcites{book,misc}{{Books},{Others}}
%----------------------------------------------------------------------------------
%            content
%----------------------------------------------------------------------------------
\begin{document}
	%\begin{CJK*}{UTF8}{gbsn}                          % to typeset your resume in Chinese using CJK
	%-----       resume       ---------------------------------------------------------
	\makecvtitle
	\vspace*{-18mm}

	%-----------------------------------------------------------------------------------------------------------
	%-----------------------------------------------------------------------------------------------------------
	\section{Research Interests}
	\vspace*{-5mm}
	\begin{multicols}{2}
	\begin{itemize}
		\item Design \& Analysis of Algorithms
		\item Graph Theory \& its applications
		\item Combinatorial Optimization
		\item Approximation \& Randomized Algorithms
		\item Complexity Theory \& Online Algorithms
		\item Social Networks Analysis
	\end{itemize}
\end{multicols}



	\section{Education}
	\cventry{Sep 2024--Present}{Master of CS. Thesis-based in Computer Science, advised by \href{https://users.encs.concordia.ca/~haruty/}{Prof. Hovhannes Harutyunyan \ExternalLink}}{\href{https://www.concordia.ca}{Concordia University \ExternalLink}}{Montreal, QC, Canada}{}{
		\begin{itemize}
			\item \textbf{GPA:} 3.53/4.0
			\item \textbf{Research Topics:} Community Detection, Social Networks Analysis, Algorithms Design, Graph Theory
%			\item \textbf{Master's Thesis:} To be decided (Community Detection on Social Networks).
%			\item \textbf{Bachelor's Thesis grade:} 20/20 (4.0/4.0)
%			\item \textbf{Supervisor:} Prof. Hovhannes Harutyunyan
				\end{itemize}} 
	\cventry{Sep 2018--Jun 2023}{B.Sc. in Computer Engineering, advised by \href{https://scholar.google.com/citations?user=UNnjGXEAAAAJ&hl=en}{Dr. Fahimeh Dabaghi-Zarandi \ExternalLink}}{\href{https://benymaxparsa.github.io/education/VRU}{Vali-e-Asr University of Rafsanjan \ExternalLink}}{Rafsanjan, Iran}{}{
		\begin{itemize}
			\item \textbf{GPA:} 16.26/20.0 \textbf{\textit{ *Graduated with Honors}}
			\item \textbf{Research Topics:} Community Detection, Algorithms Design, Machine Learning, Software Refactoring
%			\item \textbf{Bachelor's Thesis:} Community detection in complex network based on an improved random algorithm using local and global network information
%			\item \textbf{Bachelor's Thesis grade:} 20/20 (4.0/4.0)
				\end{itemize}}  % arguments 3 to 6 can be left empty

	% \section{Master thesis}
	% \cvitem{title}{\emph{Title}}

	% \cvitem{description}{Short thesis abst	% \cvitem{supervisors}{Supervisors}ract}

	%-----------------------------------------------------------------------------------------------------------
	%-----------------------------------------------------------------------------------------------------------

	\section{Publications}
%	\subsection{Journal papers}
	\begin{itemize}
		
		\item \textit{\href{https://benymaxparsa.github.io/publication/p5}{From Dense Graphs to Meaningful Communities: Assessing Community Quality Using Geodesic Distance Modularity on Metric Backbone-Sparsified Networks \ExternalLink}}
			\begin{itemize}
				\item \textbf{Parsa Kamalipour} and Hovhannes Harutyunyan [\textit{Submitted to The 12th International Conference on Social Networks Analysis, Management and Security (SNAMS 2025)}]
				\end{itemize}
		
		\item \textit{\href{https://benymaxparsa.github.io/publication/p4}{LLM-Based Code Translation for Cross-Language Refactoring Mining \ExternalLink}}
			\begin{itemize}
				\item Iman Hemati Moghadam, Mohammad Mehdi Afkhami, Vadim Zaytsev, Mohammad Hossein Ashoori, Hossein Bazmandegan, and \textbf{Parsa Kamalipour} [\textit{In Revision at Empirical Software Engineering journal (EMSE)}]
				\end{itemize}
		
		\item \textit{\href{https://benymaxparsa.github.io/publication/p3}{Extending refactoring detection to Kotlin: A dataset and comparative study \ExternalLink}}
			\begin{itemize}
				\item Iman Hemati Moghadam, Mohammad Mehdi Afkhami, \textbf{Parsa Kamalipour}, and Vadim Zaytsev [\textit{The 31st IEEE International Conference on Software Analysis, Evolution and Reengineering (SANER 2024)}, \href{http://dx.doi.org/10.1109/SANER60148.2024.00034}{doi.\ExternalLink}]
				\end{itemize}
				
		
		\item \textit{\href{https://benymaxparsa.github.io/publication/p1}{Community detection in complex network based on an improved random algorithm using local and global network information \ExternalLink}}
			\begin{itemize}
				\item Fahimeh Dabaghi-Zarandi, \textbf{Parsa Kamalipour} [\textit{Journal of Network and Computer Applications (JNCA)}, vol.206, p.103492, Aug 2022, \href{https://doi.org/10.1016/j.jnca.2022.103492}{doi.\ExternalLink}]
				\end{itemize}
				

	\end{itemize}

%$\filledstar$ \textit{Several "Machine Learning for Software Refactoring" papers are being written to be submitted in upcoming conferences.}

	%-----------------------------------------------------------------------------------------------------------
	%-----------------------------------------------------------------------------------------------------------

%	\section{Research Experience}
%	
%	\cventry{Aug 2024--Present}{Algorithms \& Complexity Lab - Department of CS \& SE, \href{https://www.concordia.ca/ginacody/computer-science-software-eng.html}{Concordia University \ExternalLink}}{
%		\href{https://benymaxparsa.github.io/research/UT-RA}{Graduate Research Assistant \ExternalLink}
%	}{Montreal, QC, Canada}{}{
%		\begin{itemize}
%			\item \textbf{Field of Research:} Algorithms Design, Graph Theory, and Social Network Analysis
%			\item \textbf{Supervisor:} Prof. Hovhannes Harutyunyan
%%			\item \textbf{My key role consisted of: }
%%			\begin{itemize}
%%				\item Writing Java codes, unit tests, debugging, refactoring, maintenance, and bug fixing the “KotlinCode2Text” parser + “RefDetect” tools
%%				\item Implementing the “KotlinCode2Text” parser for the “RefDetect” tool
%%				\item Creating two refactoring datasets
%%				\item Running Numerous testing stages and providing new ideas to improve our research results
%%				\item Prompt engineering and utilizing LLMs for Software Translation
%%			\end{itemize}
%	\end{itemize}}
%
%\vspace*{-4mm}
%		\subsection{}
%
%
%	
%	\cventry{Aug 2023--Mar 2024}{FMT group - Faculty of EE, Math and CS, \href{https://www.utwente.nl/en/}{University of Twente \ExternalLink}}{
%		\href{https://benymaxparsa.github.io/research/UT-RA}{Research Assistant (Internship) \ExternalLink}
%	}{Enschede, The Netherlands}{}{
%		\begin{itemize}
%			\item \textbf{Field of Research:} Software Refactoring, \textbf{Supervisor:} Dr. Iman Hemati Moghadam
%			\item \textbf{My key role consisted of: }
%			\begin{itemize}
%				\item Writing Java codes, unit tests, debugging, refactoring, maintenance, bug fixing the “KotlinCode2Text” parser + “RefDetect” tools, and creating two refactoring datasets.
%				\item Implementing the “KotlinCode2Text” parser for the “RefDetect” tool
%%				\item Creating two refactoring datasets
%				\item Running Numerous testing stages and providing new ideas to improve our research results
%				\item Prompt engineering and utilizing LLMs for Software Translation
%			\end{itemize}
%	\end{itemize}}
%	
%	\vspace*{-4mm}
%
%			\subsection{}
%
%	
%	\cventry{Aug 2021--Mar 2024}{Department of Computer Engineering, Vali-e-Asr University of Rafsanjan}{
%		\href{https://benymaxparsa.github.io/research/Vru-RA}{Undergraduate Research Assistant \ExternalLink}
%	}{Rafsanjan, Iran}{}{
%		\begin{itemize}
%			\item \textbf{Field of Research:} Graph Algorithms, \textbf{Supervisor:} Dr. Fahimeh Dabaghi-Zarandi.
%			\item \textbf{My key role consisted of: }
%			\begin{itemize}
%				\item Reading and reviewing related papers (Investigation)
%				\item Implementing ideas in MATLAB and Python (Data curation, Software, Programming)
%				\item Testing and improving the written code (Validation)
%				\item Gathering information and writing the initial text for the paper (Writing primary draft preparation)
%			\end{itemize}
%	\end{itemize}}
%	%-----------------------------------------------------------------------------------------------------------
\section{Experiences}
	\subsection{Research Experience}

\cventry{Aug 2024 -- Present}
  {Graduate Research Assistant}
  {\href{https://www.concordia.ca/ginacody/computer-science-software-eng.html}{Algorithms \& Complexity Lab, Concordia University}}
  {Montreal, QC, Canada}
  {Supervisor: Prof. Hovhannes Harutyunyan}
  {\begin{itemize}%
	\item Conducting research in \textbf{algorithm design, graph theory, and social network analysis}, with emphasis on large-scale social networks.  
    \item Investigating \textbf{Geodesic Distance Metric and parameter-free sparsification} methods to evaluate and enhance community detection.  
    \item Developing new \textbf{theoretical frameworks and algorithms} to advance the study of community quality in networks.  
  \end{itemize}}

\vspace*{-3mm}
\subsection{}

\cventry{Aug 2023 -- Mar 2024}
  {Research Collaborator (Remote)}
  {\href{https://www.utwente.nl/en/}{Formal Methods and Tools (FMT) Group, University of Twente}}
  {Enschede, The Netherlands}
  {Supervisor: Dr. Iman Hemati Moghadam}
  {\begin{itemize}%
    \item Implemented the \textbf{“KotlinCode2Text” parser} and integrated it into the \textbf{“RefDetect” tool} for automated refactoring detection.
    \item Built two refactoring datasets supporting empirical evaluation of refactoring detection techniques.
    \item Improved tool performance via systematic testing, debugging, and algorithmic optimizations.
    \item Explored \textbf{prompt engineering with large language models (LLMs)} to enhance software translation tasks.
  \end{itemize}}

\vspace*{-3mm}
\subsection{}

\cventry{Aug 2021 -- Mar 2024}
  {Undergraduate Research Assistant}
  {\href{https://benymaxparsa.github.io/research/Vru-RA}{Department of Computer Engineering, Vali-e-Asr University of Rafsanjan}}
  {Rafsanjan, Iran}
  {Supervisor: Dr. Fahimeh Dabaghi-Zarandi}
  {\begin{itemize}%
    \item Investigated algorithmic approaches for solving complex problems in \textbf{graph theory}.
    \item Implemented and validated graph algorithms in \textbf{MATLAB and Python}.
    \item Curated datasets and evaluated algorithmic performance through experimental studies.
    \item Drafted preliminary manuscripts and contributed to research publications.
  \end{itemize}}
	%-----------------------------------------------------------------------------------------------------------
%	\section{Teaching Experience}
%	
%	\cventry{Sep 2024--Present}{Gina Cody School of Engineering and Computer Science, \href{https://www.concordia.ca/ginacody.html}{Concordia University \ExternalLink}}{Graduate Teaching Assistant}{Montreal, QC, Canada}{}{}
%
%		\cvitemwithcomment{COMP 233: Probability and Statistics for Computer Science}{Tutorial leader: Summer 2 \href{https://benymaxparsa.github.io/teaching/Summer2-2025-COMP233}{2025 \ExternalLink}}{}	
%
%		\cvitemwithcomment{COMP 348: Principles of Programming Languages}{Tutorial leader \& Marker: Summer 1 \href{https://benymaxparsa.github.io/teaching/Summer1-2025-COMP348}{2025 \ExternalLink}}{}	
%
%		\cvitemwithcomment{COMP 335: Introduction to Theoretical Computer Science}{Marker: Summer 1 \href{https://benymaxparsa.github.io/teaching/Summer1-2025-COMP335}{2025 \ExternalLink}}{}		
%		
%		\cvitemwithcomment{COMP 348: Principles of Programming Languages}{Tutorial leader \& POD \& Marker: Winter \href{https://benymaxparsa.github.io/teaching/Winter-2025-COMP348}{2025 \ExternalLink}}{}
%		
%		\cvitemwithcomment{COMP 465: Design and Analysis of Algorithms}{Tutorial leader: Winter \href{https://benymaxparsa.github.io/teaching/Winter-2025-COMP465}{2025 \ExternalLink}}{}
%		
%		\cvitemwithcomment{SOEN 363: Data Systems for Software Engineers}{Tutorial leader \& POD \& Marker: Winter \href{https://benymaxparsa.github.io/teaching/Winter-2025-SOEN363}{2025 \ExternalLink}}{}
%		
%		\cvitemwithcomment{COMP/MATH 339: Combinatorics}{Tutorial leader \& Marker TA: Fall \href{https://benymaxparsa.github.io/teaching/Fall-2024-COMP339}{2024 \ExternalLink}}{}
%
%		\cvitemwithcomment{COMP 335: Introduction to Theoretical Computer Science}{Marker TA: Fall \href{https://benymaxparsa.github.io/teaching/Fall-2024-COMP335}{2024 \ExternalLink}}{}
%
%	\vspace*{-4mm}
%		\subsection{}
%		
%	\cventry{Mar 2021--Jan 2024}{CE Department, Vali-e-Asr University of Rafsanjan}{Undergraduate Teaching Assistant}{Rafsanjan, Iran}{}{}
%%	\vspace*{-1.6mm}
%
%%	\subsection{Courses of Dr. Fahimeh Dabaghi-Zarandi}
%		\cvitemwithcomment{Data Structures}{Head TA: Spring (\href{https://benymaxparsa.github.io/teaching/Spring-2023-DS}{2023 \ExternalLink}, \href{https://benymaxparsa.github.io/teaching/Spring-2022-DS}{2022 \ExternalLink}, \href{https://benymaxparsa.github.io/teaching/Spring-2021-DS}{2021 \ExternalLink), Fall (\href{https://benymaxparsa.github.io/teaching/Fall-2023-DS}{2023 \ExternalLink},\href{https://benymaxparsa.github.io/teaching/Fall-2022-DS}{2022 \ExternalLink},\href{https://benymaxparsa.github.io/teaching/Fall-2021-DS}{2021 \ExternalLink})}}{}
%		\cvitemwithcomment{Algorithms Design}{Head TA: Spring (\href{https://benymaxparsa.github.io/teaching/Spring-2023-DA}{2023 \ExternalLink},\href{https://benymaxparsa.github.io/teaching/Spring-2022-DA}{2022 \ExternalLink}, \href{https://benymaxparsa.github.io/teaching/Spring-2021-DA}{2021 \ExternalLink}), Fall (\href{https://benymaxparsa.github.io/teaching/Fall-2022-DA}{2022 \ExternalLink}, \href{https://benymaxparsa.github.io/teaching/Fall-2021-DA}{2021 \ExternalLink})}{}
%		\cvitemwithcomment{Discrete Mathematics}{TA: \href{https://benymaxparsa.github.io/teaching/Spring-2022-DM}{Spring 2022 \ExternalLink}, \href{https://benymaxparsa.github.io/teaching/Fall-2021-DM}{Fall 2021 \ExternalLink}}{}
%		\cvitemwithcomment{Operating Systems}{Grading TA: \href{https://benymaxparsa.github.io/teaching/Spring-2022-OS}{Spring 2022 \ExternalLink}}{}
%
%%	\subsection{Courses of Dr. Mojtaba Sabbagh-Jafari}
%		\cvitemwithcomment{Introduction to Information Retrieval}{TA: Spring (\href{https://benymaxparsa.github.io/teaching/Spring-2023-IR}{2023 \ExternalLink}, \href{https://benymaxparsa.github.io/teaching/Spring-2022-IR}{2022 \ExternalLink})}{}
%		\cvitemwithcomment{Software Engineering}{TA: \href{https://benymaxparsa.github.io/teaching/Spring-2023-SE}{Spring 2023 \ExternalLink}}{}
%		\cvitemwithcomment{Database}{Head TA: \href{https://benymaxparsa.github.io/teaching/Fall-2022-DB}{Fall 2022 \ExternalLink}}{}
%
%%	\subsection{Courses of Dr. Amir Hossein Hadjahmadi}
%		\cvitemwithcomment{Intro to Data Mining}{TA: \href{https://benymaxparsa.github.io/teaching/Spring-2023-DTM}{Spring 2023 \ExternalLink}}{}
%		\cvitemwithcomment{Fundamentals of Programming}{Head TA: \href{https://benymaxparsa.github.io/teaching/Fall-2022-FP}{Fall 2022 \ExternalLink}}{}
%		\cvitemwithcomment{Artificial Intelligence}{Head TA: \href{https://benymaxparsa.github.io/teaching/Fall-2022-AI}{Fall 2022 \ExternalLink}}{}
%
%
%%\cventry{Summer 2022}{Vali-e-Asr University Scientific Association of Computer Engineering}{\href{https://benymaxparsa.github.io/teaching/Summer-2022-SCB}{Python \& Git \& Github Instructor for The Summer Coding Bootcamp \ExternalLink}}{Rafsanjan, Iran}{}{}
%
%%\cventry{Summer 2022}{Freelance}{\href{https://benymaxparsa.github.io/teaching/Summer-2022-python}{Python Private Tutor \ExternalLink}}{Kerman, Iran}{}{}
%

\subsection{Teaching Experience}

\cventry{Sep 2024 -- Present}
  {Graduate Teaching Assistant}
  {\href{https://www.concordia.ca/ginacody.html}{Gina Cody School of Engineering and Computer Science, Concordia University}}
  {Montreal, QC, Canada}
  {}
  {\begin{itemize}%
    \item Led tutorials, graded assignments and exams for core undergraduate courses, including \textbf{Algorithms, Programming Languages, and Data Systems}.
    \item Supported student learning through \textbf{Programmer On Duty (POD)} [Q\&A Sessions], assignment guidance, and evaluation.
    \item Courses:  
      \begin{itemize}%
        \item COMP 233: Probability and Statistics for Computer Science (Summer 2025)  
        \item COMP 348: Principles of Programming Languages (Winter 2025, Summer 2025)  
        \item COMP 465: Design and Analysis of Algorithms (Winter 2025)  
        \item SOEN 363: Data Systems for Software Engineers (Winter 2025)  
        \item COMP/MATH 339: Combinatorics (Fall 2024)  
        \item COMP 335: Introduction to Theoretical Computer Science (Fall 2024, Summer 2025)  
      \end{itemize}
  \end{itemize}}

\vspace*{-2mm}
\subsection{}

\cventry{Mar 2021 -- Jan 2024}
  {Undergraduate Teaching Assistant}
  {Department of Computer Engineering, Vali-e-Asr University of Rafsanjan}
  {Rafsanjan, Iran}
  {}
  {\begin{itemize}%
    \item Served as \textbf{Head TA and Tutorial Leader} for multiple foundational CS courses, mentoring students and overseeing grading.
    \item Collaborated with faculty to design assignments, run labs, and support student projects in algorithms, data structures, and software engineering.
    \item Courses:  \vspace{-\baselineskip}
      \begin{multicols}{2}
        \begin{itemize}
          \item Data Structures (Spring 2021–2023, Fall 2021–2023)
          \item Algorithms Design (Spring 2021–2023, Fall 2021–2022)
          \item Discrete Mathematics (Fall 2021, Spring 2022)
          \item Operating Systems (Spring 2022)
          \item Introduction to Information Retrieval (Spring 2022–2023)
          \item Software Engineering (Spring 2023)
          \item Database Systems (Fall 2022)
          \item Fundamentals of Programming (Fall 2022)
          \item Artificial Intelligence (Fall 2022)
          \item Introduction to Data Mining (Spring 2023)
        \end{itemize}
      \end{multicols}
  \end{itemize}}

%\cventry{Mar 2021 -- Jan 2024}
%  {Undergraduate Teaching Assistant}
%  {Department of Computer Engineering, Vali-e-Asr University of Rafsanjan}
%  {Rafsanjan, Iran}
%  {}
%  {\begin{itemize}%
%    \item Served as \textbf{Head TA and Tutorial Leader} for multiple foundational CS courses, mentoring students and overseeing grading.
%    \item Collaborated with faculty to design assignments, run labs, and support student projects in algorithms, data structures, and software engineering.
%    \item Courses:  
%      \begin{itemize}%
%        \item Data Structures (Spring 2021–2023, Fall 2021–2023)  
%        \item Algorithms Design (Spring 2021–2023, Fall 2021–2022)  
%        \item Discrete Mathematics (Fall 2021, Spring 2022)  
%        \item Operating Systems (Spring 2022)  
%        \item Introduction to Information Retrieval (Spring 2022–2023)  
%        \item Software Engineering (Spring 2023)  
%        \item Database Systems (Fall 2022)  
%        \item Fundamentals of Programming (Fall 2022)  
%        \item Artificial Intelligence (Fall 2022)  
%        \item Introduction to Data Mining (Spring 2023)  
%      \end{itemize}
%  \end{itemize}}
%-----------------------------------------------------------------------------------------------------------
%-----------------------------------------------------------------------------------------------------------
\subsection{Industry Experience}

\cventry{Feb 2020 -- Sep 2021}
  {Team Co-Founder \& Indie Game Developer}
  {\href{https://github.com/NullReferences}{Null References: Game Development Team}}
  {Kerman, Iran}
  {}
  {\begin{itemize}%
    \item Co-founded an indie game development team, collaborating on all stages of game design and implementation.  
    \item Applied \textbf{Design Patterns} and \textbf{SOLID principles} to develop a demo of the video game \href{https://benymaxparsa.github.io/industry/}{\textit{Uncertainty}}.  
    \item Released the project as an \textbf{open-source game} on GitHub, contributing to community-driven development.  
  \end{itemize}}

%-----------------------------------------------------------------------------------------------------------
%-----------------------------------------------------------------------------------------------------------

%-----------------------------------------------------------------------------------------------------------
%-----------------------------------------------------------------------------------------------------------
%
%\section{Selected Relevant Coursework}
%\cvdoubleitem{Fundamental of Programming}{20/20}{Advance Programming}{16.5/20}
%\cvdoubleitem{Theory of Machines and Languages}{17.9/20}{Engineering Mathematics}{17.04/20}
%\cvdoubleitem{Software Engineering}{18.75/20}{Design and Analysis of Algorithms}{18/20}
%%\cvdoubleitem{Digital Logic Design}{19.45/20}
%\cvdoubleitem{Introduction to Data Mining}{17.5/20}{Statistics and Probability for Engineering}{16.25/20}
%%\cvdoubleitem{Programming Language Design}{18/20}{Database}{16/20}
%%	\cvdoubleitem{Theory of Machines \& Languages}{17.9/20}{The principles of Compiler Design}{20/20}
%$\filledstar$ \textit{\href{https://benymaxparsa.github.io/selected_courses}{Click here to see more \ExternalLink}}


%-----------------------------------------------------------------------------------------------------------
%-----------------------------------------------------------------------------------------------------------

\section{Honors and Awards}

\cvitem{2024}{\textbf{DRW Graduate Scholarship in Computer Science} -- Concordia University \& DRW Company}
\cvitem{2024}{\textbf{Concordia Merit Scholarship (Entrance Scholarship Award)} -- Concordia University, School of Graduate Studies}
\cvitem{2024}{\textbf{Financial Research Support (FRS)} -- Concordia Faculty of Engineering and Computer Science}
\cvitem{2023}{\textbf{Distinguished Student Award} -- Awarded among all students of Vali-e-Asr University}
\cvitem{2023}{\textbf{Undergraduate Researcher Award} -- Awarded among all undergraduate students of Vali-e-Asr University}
\cvitem{2023}{\textbf{Top Researcher Award} -- Earned this prestige award among all undergraduate students of Kerman Province}
%$\filledstar$ \textit{\href{https://benymaxparsa.github.io/honors_and_extra/}{Click here to see more information about me \ExternalLink}}

%-----------------------------------------------------------------------------------------------------------
%-----------------------------------------------------------------------------------------------------------

\section{Selected Projects}

\cventry{Spring 2022}{Multiple assignments regarding to the Intro to Data Mining course}{\href{https://benymaxparsa.github.io/projects/DTM-Projects}
	{Introduction to Data Mining \ExternalLink}}{}{}{
		Data Pre Processing, Apriori Algorithm, Data Visualization, K-Means, Agglomerative Clustering, DBSCAN, K-Nearest Neighbors Algorithm, Decision Tree, Support Vector Machines, Multi-Layer Perceptron
	}
%
%\cventry{Fall 2021}{Designing and implementation of:}{\href{https://benymaxparsa.github.io/projects/AI-Projects/}{Multiple projects regarding to Artificial Intelligence course \ExternalLink}}{}{}{
%		BFS, DFS, IDS, UCS (Uninformed Search Strategies),
%		8 Puzzle solver using A-star \& IDA (Informed Heuristic Search Strategies),
%		genetic algorithms,
%		simulated annealing (Local Search),
%		Min-Max, Alpha–Beta (Adversarial Search),
%		classification of a dataset (Basic Machine Learning),
%		knowledge representation using prolog
%	}
%
%\cventry{Fall 2021}{One project regarding to the Database course}{\href{https://benymaxparsa.github.io/projects/SYMPHONYC}
%	{SYMPHONYC: The database of a music streaming service similar to Spotify. \ExternalLink}}{}{}{
%		Information Gathering, entity-relationship model, Relational Model, SQL codes, connecting the database to Django, analyzing and plotting data via matplotlib
%	}


\cventry{Spring 2021}{\href{https://github.com/Null-References}{Null References \ExternalLink}}{\href{https://benymaxparsa.github.io/projects/Uncertainty}{Uncertainty: an action-adventure space-shooter game built with Unity3D \ExternalLink}}{}{}{
	\begin{itemize}
		\item  Uncertainty is an action-adventure space-shooter game, and currently It’s under development.
		\item We have utilized the beta version of this game as our ”Software Engineering Lab” course project.
	\end{itemize}
}

%\cventry{Spring 2021}{Projects:}{\href{https://benymaxparsa.github.io/projects/IR-Projects/}
%	{Two projects regarding to Introduction to Information Retrieval course \ExternalLink}}{}{}{The Scrapy Crawler (Crawling), Inverted Index Construction using BSBI Algorithm (Indexing)}

\cventry{Fall 2020}{Designing and implementation of: }{\href{https://benymaxparsa.github.io/projects/DA-Projects}
	{Multiple projects regarding to Design and Analysis of Algorithms course \ExternalLink}}{}{}{The Closest Pair of Points Problem, Sudoku Solver, Tournament Scheduler, Huffman Coding, Bellman–Ford, Matrix Chain Multiplication, N-Queens Solver Traveling Salesman Problem}

\cventry{Fall 2019}{Designing and implementation of:}{
	\href{https://benymaxparsa.github.io/projects/DS-Projects/}
	{Multiple projects regarding to Data Structures and Algorithms course \ExternalLink}
}{}{}{
	the Red-Black Tree, the AVL Tree, the Trie Dictionary, Threaded Binary Tree, the Sparse Matrix via Linked List, the Rat in the maze problem
}

%$\filledstar$ \textit{\href{https://benymaxparsa.github.io/projects}{Click here to see more projects \ExternalLink}}

%-----------------------------------------------------------------------------------------------------------
%-----------------------------------------------------------------------------------------------------------
%
%\section{Test Scores}
%\cvitemwithcomment{TOEFL}{99/120 - Reading: 26/30, Listening: 29/30, Speaking: 23/30, Writing: 21/30}{}
%%\cvitemwithcomment{GRE}{Not taken}{}

%-----------------------------------------------------------------------------------------------------------
%-----------------------------------------------------------------------------------------------------------
%
%\section{Honors and Awards}
%
%\cventry{Oct 2024}{Issued by Concordia University \& DRW Company}{Awarded the DRW Graduate Scholarship in Computer Science}{}{}{}
%
%\cventry{Sep 2024}{Issued by Concordia University \& School of Graduate Studies}{Awarded Concordia Merit Scholarship (Entrance Scholarship Award)}{}{}{}
%
%\cventry{Sep 2024}{Issued by Concordia University \& Gina Cody School of Engineering and Computer Science}{Awarded Concordia Faculty of Engineering \& Computer Science Financial Research Support}{}{}{}
%
%%\cventry{July 2024}{Vali-e-Asr University of Rafsanjan}{Won Distinguished Student Award among all students of Vali-e-Asr University}{}{}{}
%
%%	\cventry{Dec 2023}{Vali-e-Asr University of Rafsanjan}{Awarded Undergraduate Researcher status among all students of Vali-e-Asr University}{}{}{}
%	\cventry{Nov 2023}{Shahid Bahonar University of Kerman}{Awarded Top Researcher status among all undergraduate students of Kerman province}{}{}{}
%%	\cventry{Spring 2023}{for ICPC Asia Tehran - Internet Programming Contest}{Ranked 1st among all teams participating from the Vali-e-Asr University}{}{}{}
%%	\cventry{Spring 2022}{Vali-e-Asr University of Rafsanjan}{Ranked top 10 among undergraduate students of Computer Engineering}{}{Among Fall of 2018 Computer Engineering students}{}
%
%%\cventry{}{Vali-e-Asr University of Rafsanjan}{Awarded by government undergraduate tuition waiver scholarship}{}{}{}
%
%$\filledstar$ \textit{\href{https://benymaxparsa.github.io/honors_and_extra/}{Click here to see more information about me \ExternalLink}}

%-----------------------------------------------------------------------------------------------------------
%-----------------------------------------------------------------------------------------------------------

%\section{Extra Curricular Activities}
%\cventry{Nov 2022--Sep 2023}{Vali-e-Asr University Scientific Association of Computer Engineering}{\href{https://benymaxparsa.github.io/extracurricular/ra_committee}{Director of Research Assistant Committee \ExternalLink}}{}{}{}
%\cventry{July 2022--Sep 2023}{Vali-e-Asr University Scientific Association of Computer Engineering}{\href{https://benymaxparsa.github.io/extracurricular/ta_committee}{Director of Teaching Assistant Committee \ExternalLink}}{}{}{}
%\cventry{Sep 2021--Jun 2022}{Vali-e-Asr Collegiate Programming Contest (VCPC)}{\href{https://benymaxparsa.github.io/extracurricular/VCPC_member}{Member of Teaching Staff \ExternalLink}}{}{}{}
%\cventry{May 2019--Jun 2021}{Vali-e-Asr University Scientific Association of Computer Engineering}{\href{https://benymaxparsa.github.io/extracurricular/CEA_member}{Scientific Committee Member \ExternalLink}}{}{}{}
%\cventry{Oct 2020--Jun 2021}{Vali-e-Asr Video Games Association}{\href{https://benymaxparsa.github.io/extracurricular/VGA_member}{Executive Committee Member \ExternalLink}}{}{}{}




%-----------------------------------------------------------------------------------------------------------
%-----------------------------------------------------------------------------------------------------------

\section{Skills}

\cvdoubleitem{Programming Languages}{C, C++, Python, MATLAB, C\#, Java, SQL}
              {Frameworks \& Libraries}{Qt, NumPy, Pandas, Matplotlib, NetworkX, Scikit-learn, PyTorch, Unity}

\cvdoubleitem{Algorithms \& Data Science}{Graph Algorithms, Community Detection, Social Network Analysis, Machine Learning}
              {Software Engineering}{Refactoring, Debugging, Unit Testing, Agile Methodologies, Design Patterns, SOLID Principles}

\cvdoubleitem{Tools \& Platforms}{Linux, Git, Jupyter, \LaTeX, Markdown, Microsoft Office, Obsidian}
              {Soft Skills}{Teamwork, Leadership, Collaboration, Teaching, Research, Problem Solving}

%$\filledstar$ \textit{\href{https://www.linkedin.com/in/parsakamalipour/details/skills/}{Click here to see more in LinkedIn \ExternalLink}}



\section{Selected Relevant Coursework}

\cvitem{Graduate}{Algorithm Design Techniques, Advanced Analysis of Algorithms, Combinatorial Algorithms, Machine Learning}

\cvitem{Undergraduate}{Design and Analysis of Algorithms, Data Structures, Discrete Mathematics, Programming Language Design, Artificial Intelligence, Software Engineering, Fundamentals of Data Mining, Compiler Design, Operating Systems, Computer Architecture}
%\cvdoubleitem{Programming Language Design}{18/20}{Database}{16/20}
%	\cvdoubleitem{Theory of Machines \& Languages}{17.9/20}{The principles of Compiler Design}{20/20}
%$\filledstar$ \textit{\href{https://benymaxparsa.github.io/selected_courses}{Click here to see more \ExternalLink}}

%
\section{Test Scores}
\cvitemwithcomment{TOEFL}{99/120 - Reading: 26/30, Listening: 29/30, Speaking: 23/30, Writing: 21/30}{}
%\cvitemwithcomment{GRE}{Not taken}{}
%-----------------------------------------------------------------------------------------------------------
%-----------------------------------------------------------------------------------------------------------

\section{Languages}

\begin{tabular*}{\textwidth}{@{\extracolsep{\fill}} l l l}
\textbf{Persian}: Native & \textbf{English}: Proficient & \textbf{French}: Beginner (A1) \\
\end{tabular*}


\section{Volunteer Experience}

\cventry{Nov 2022 -- Sep 202it3}
  {Director, Research Assistant Committee}
  {Vali-e-Asr University Scientific Association of Computer Engineering}
  {}
  {}
  {}

\cventry{Jul 2022 -- Sep 2023}
  {Director, Teaching Assistant Committee}
  {Vali-e-Asr University Scientific Association of Computer Engineering}
  {}
  {}
  {}

\cventry{Sep 2021 -- Jun 2022}
  {Teaching Staff Member}
  {Vali-e-Asr Collegiate Programming Contest (VCPC)}
  {}
  {}
  {}
  
  %--------------------------------------------------------------------------------------
%-----------------------------------------------------------------------------------------------------------

\section{References}

Available upon Request

% \cventry{}{Assistant Professor}{Dr. Fahimeh Dabaghi-Zarandi}{Rafsanjan, Iran}{\href{mailto:f.dabaghi@vru.ac.ir}
% 	{f.dabaghi@vru.ac.ir}}{Department of Computer Engineering, Faculty of Engineering, Vali-e-Asr University of Rafsanjan}

% \cventry{}{Assistant Professor}{Dr. Mojtaba Sabbagh-Jafari}{Rafsanjan, Iran}{\href{mailto:mojtaba.sabbagh@vru.ac.ir}
% 	{mojtaba.sabbagh@vru.ac.ir}}{Department of Computer Engineering, Faculty of Engineering, Vali-e-Asr University of Rafsanjan}

% %-----------------------------------------------------------------------------------------------------------
% %-----------------------------------------------------------------------------------------------------------

% \cventry{}{Assistant Professor}{Dr. Amir Hossein Hadjahmadi}{Rafsanjan, Iran}{\href{mailto:hadjahmadi@vru.ac.ir}
% 	{hadjahmadi@vru.ac.ir}}{Department of Computer Engineering, Faculty of Engineering, Vali-e-Asr University of Rafsanjan}

%	\null\hfill%
%\hfill[\textit{\CVNote}]

% \section{Skill matrix}
% \cvitem{Skill matrix}{Alternatively, provide a skill matrix to show off your skills}
% %% Skill matrix as an alternative to rate one's skills, computer or other.

%% Adjusts width of skill matrix columns.
%% Usage \setcvskillcolumns[<width>][<factor>][<exp_width>]
%% <width>, <exp_width> should be lengths smaller than \textwidth, <factor> needs to be between 0 and 1.
%% Examples:
% \setcvskillcolumns[5em][][]%    adjust first column. Same as \setcvskillcolumns[5em]
% \setcvskillcolumns[][0.45][]%   adjust third (skill) column. Same as \setcvskillcolumns[][0.45]
% \setcvskillcolumns[][][\widthof{``Year''}]%     adjust fourth (years) column.
% \setcvskillcolumns[][0.45][\widthof{``Year''}]%
% \setcvskillcolumns[\widthof{``Languag''}][0.48][]
% \setcvskillcolumns[\widthof{``Languag''}]%

%% Adjusts width of legend columns. Usage \setcvskilllegendcolumns[<width>][<factor>]
%% <factor> needs to be between 0 and 1. <width> should be a length smaller than \textwidth
%% Examples:
% \setcvskilllegendcolumns[][0.45]
% \setcvskilllegendcolumns[\widthof{``Legend''}][0.45]
% \setcvskilllegendcolumns[0ex][0.46]% this is usefull for the banking style

%% Add a legend if you are using \cvskill{<1-5>} command or \cvskillentry
%% Usage \cvskilllegend[*][<post_padding>][<first_level>][<second_level>][<third_level>][<fourth_level>][<fifth_level>]{<name>}
% \cvskilllegend % insert default legend without lines
% \cvskilllegend*[1em]{}% adjust post spacing
% \cvskilllegend*{Legend}%  Alternatively add a description string
%% adjust the legend entries for other languages, here German
% \cvskilllegend[0.2em][Grundkenntnisse][Grundkenntnisse und eigene Erfahrung in Projekten][Umfangreiche Erfahrung in Projekten][Vertiefte Expertenkenntnisse][Experte\,/\,Spezialist]{Legende}

%% Alternative legend style with the first three skill levels in one column
%% Usage \cvskillplainlegend[*][<post_padding>][<first_level>][<second_level>][<third_level>][<fourth_level>][<fifth_level>]{<name>}
% \setcvskilllegendcolumns[][0.6]%  works for classic, casual, banking
% \setcvskilllegendcolumns[][0.55]%  works better for oldstyle and fancy
% \cvskillplainlegend{}
% \cvskillplainlegend[0.2em][Grundkenntnisse][Grundkenntnisse und eigene Erfahrung in Projekten][Umfangreiche Erfahrung in Projekten][Vertiefte Expertenkenntnisse][Experte/Guru]{Legende}

%% Add a head of the skill matrix table with descriptions.
%% Usage \cvskillhead[<post_padding>][<Level>][<Skill>][<Years>][<Comment>]%
% \cvskillhead[-0.1em]%   this inserts the standard legend in english and adjust padding
%% Adjust head of the skill matrix for other languages
% \cvskillhead[0.25em][Level][F\"ahigkeit][Jahre][Bemerkung]

%% \cvskillentry[*][<post_padding>]{<skill_cathegory>}{<0-5>}{<skill_name>}{<years_of_experience>}{<comment>}%
%% Example usages:
% \cvskillentry*{Language:}{3}{Python}{2}{I'm so experienced in Python and have realised a million projects. At least.}
% \cvskillentry{}{2}{Lilypond}{14}{So much sheet music! Man, I'm the best!}
% \cvskillentry{}{3}{\LaTeX}{14}{Clearly I rock at \LaTeX}
% \cvskillentry*{OS:}{3}{Linux}{2}{I only use Archlinux btw}% notice the use of the starred command and the optional
% \cvskillentry*[1em]{Methods}{4}{SCRUM}{8}{SCRUM master for 5 years}
%% \cvskill{<0-5>} command
% \cvitem{\textbackslash{cvskill}:}{Skills can be visually expressed by the \textbackslash{cvskill} command, e.g. \cvskill{2}}

% \section{Interests}
% \cvitem{hobby 1}{Description}
% \cvitem{hobby 2}{Description}
% \cvitem{hobby 3}{Description}

% \section{Extra 1}
% \cvlistitem{Item 1}
% \cvlistitem{Item 2}
% \cvlistitem{Item 3. This item is particularly long and therefore normally spans over several lines. Did you notice the indentation when the line wraps?}

% \section{Extra 2}
% \cvlistdoubleitem{Item 1}{Item 4}
% \cvlistdoubleitem{Item 2}{Item 5\cite{book2}}
% \cvlistdoubleitem{Item 3}{Item 6. Like item 3 in the single column list before, this item is particularly long to wrap over several lines.}

% \section{References}
% \begin{cvcolumns}
	%   \cvcolumn{Category 1}{\begin{itemize}\item Person 1\item Person 2\item Person 3\end{itemize}}
	%   \cvcolumn{Category 2}{Amongst others:\begin{itemize}\item Person 1, and\item Person 2\end{itemize}(more upon request)}
	%   \cvcolumn[0.5]{All the rest \& some more}{\textit{That} person, and \textbf{those} also (all available upon request).}
	% \end{cvcolumns}

% Publications from a BibTeX file without multibib
%  for numerical labels: \renewcommand{\bibliographyitemlabel}{\@biblabel{\arabic{enumiv}}}% CONSIDER MERGING WITH PREAMBLE PART
%  to redefine the heading string ("Publications"): \renewcommand{\refname}{Articles}
% \nocite{*}
% \bibliographystyle{plain}
% \bibliography{publications}                        % 'publications' is the name of a BibTeX file

% Publications from a BibTeX file using the multibib package
%\section{Publications}
%\nocitebook{book1,book2}
%\bibliographystylebook{plain}
%\bibliographybook{publications}                   % 'publications' is the name of a BibTeX file
%\nocitemisc{misc1,misc2,misc3}
%\bibliographystylemisc{plain}
%\bibliographymisc{publications}                   % 'publications' is the name of a BibTeX file

% \clearpage
% %-----       letter       ---------------------------------------------------------
% % recipient data
% \recipient{Company Recruitment team}{Company, Inc.\\123 somestreet\\some city}
% \date{January 01, 1984}
% \opening{Dear Sir or Madam,}
% \closing{Yours faithfully,}
% \enclosure[Attached]{curriculum vit\ae{}}          % use an optional argument to use a string other than "Enclosure", or redefine \enclname
% \makelettertitle

% Lorem ipsum dolor sit amet, consectetur adipiscing elit. Duis ullamcorper neque sit amet lectus facilisis sed luctus nisl iaculis. Vivamus at neque arcu, sed tempor quam. Curabitur pharetra tincidunt tincidunt. Morbi volutpat feugiat mauris, quis tempor neque vehicula volutpat. Duis tristique justo vel massa fermentum accumsan. Mauris ante elit, feugiat vestibulum tempor eget, eleifend ac ipsum. Donec scelerisque lobortis ipsum eu vestibulum. Pellentesque vel massa at felis accumsan rhoncus.

% Suspendisse commodo, massa eu congue tincidunt, elit mauris pellentesque orci, cursus tempor odio nisl euismod augue. Aliquam adipiscing nibh ut odio sodales et pulvinar tortor laoreet. Mauris a accumsan ligula. Class aptent taciti sociosqu ad litora torquent per conubia nostra, per inceptos himenaeos. Suspendisse vulputate sem vehicula ipsum varius nec tempus dui dapibus. Phasellus et est urna, ut auctor erat. Sed tincidunt odio id odio aliquam mattis. Donec sapien nulla, feugiat eget adipiscing sit amet, lacinia ut dolor. Phasellus tincidunt, leo a fringilla consectetur, felis diam aliquam urna, vitae aliquet lectus orci nec velit. Vivamus dapibus varius blandit.

% Duis sit amet magna ante, at sodales diam. Aenean consectetur porta risus et sagittis. Ut interdum, enim varius pellentesque tincidunt, magna libero sodales tortor, ut fermentum nunc metus a ante. Vivamus odio leo, tincidunt eu luctus ut, sollicitudin sit amet metus. Nunc sed orci lectus. Ut sodales magna sed velit volutpat sit amet pulvinar diam venenatis.

% Albert Einstein discovered that $e=mc^2$ in 1905.

% \[ e=\lim_{n \to \infty} \left(1+\frac{1}{n}\right)^n \]

% \makeletterclosing

%\clearpage\end{CJK*}                              % if you are typesetting your resume in Chinese using CJK; the \clearpage is required for fancyhdr to work correctly with CJK, though it kills the page numbering by making \lastpage undefined
\end{document}


%% end of file `template.tex'.

